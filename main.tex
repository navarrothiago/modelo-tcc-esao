% TODO: ATENÇÃO: FALTA IMPLEMENTAR OS ELEMENTOS OPCIONAIS.
% Se usar os comandos de gerar esses elementos, não estará formatado conforme a MATAD 2017 (http://www.esao.eb.mil.br/images/Arquivos/spg/material/MATAD_2017.pdf).

\documentclass[12pt,a4paper]{article}
\usepackage[margin=2cm, top=3cm, bottom=3cm]{geometry}
\usepackage[T1]{fontenc}
\usepackage[scaled]{helvet}
\renewcommand*\familydefault{\sfdefault}
\usepackage[utf8]{inputenc}
\usepackage[portuguese]{babel}
\usepackage{setspace}
\usepackage{lipsum} % para gerar texto fictício
\usepackage{graphicx}
\usepackage{setspace}
\usepackage{titlesec} % pacote para personalizar títulos de seções
\usepackage{lipsum} % para gerar texto fictício

\usepackage{tocloft}

% Useful packages
\usepackage{amsmath}
\usepackage{graphicx}
\usepackage[colorlinks=false, allcolors=blue]{hyperref}


\def\mytitle{MEU TÍTULO}
\def\myname{Cap QEM MEU NOME COMPLETO}
\def\myadvisor{Maj QEM Fulano de \textbf{Tal}}

\titleformat*{\section}{\bfseries\large\MakeUppercase} % configura a formatação do título da seção

\titleformat*{\subsection}{\large\MakeUppercase} % configura a formatação do subtítulo da seção

\titleformat*{\subsubsection}{\bfseries\large\MakeUppercase} % configura a formatação do subtítulo da seção

\titlespacing*{\section} {0pt}{2\baselineskip}{2\baselineskip} % configura o espaçamento do título da seção
\titlespacing*{\subsection} {0pt}{2\baselineskip}{2\baselineskip} % configura o espaçamento do título da seção
\titlespacing*{\subsubsection} {0pt}{2\baselineskip}{2\baselineskip} % configura o espaçamento do título da seção

\renewcommand{\cftsecfont}{\bfseries} % seções em negrito
\renewcommand{\cftsubsecfont}{} % subseções sem negrito
\renewcommand{\cftsubsubsecfont}{\bfseries} % sub sub seções em negrito

\renewcommand{\cftsecleader}{\cftdotfill{\cftdotsep}} % linhas pontilhadas até a página
\renewcommand{\cftsecaftersnum}{.} % ponto após o número da seção
\renewcommand{\cftsubsecaftersnum}{.} % ponto após o número da subseção
\setlength{\cftbeforesecskip}{1em} % espaço antes da seção
\setlength{\cftbeforesubsecskip}{1em} % espaço antes da subseção
\setlength{\cftbeforesubsubsecskip}{1em} % espaço antes da subseção
\renewcommand{\cftsecafterpnum}{\vskip0pt} % sem espaço extra após o número da página da seção
\renewcommand{\cftsubsecafterpnum}{\vskip0pt} % sem espaço extra após o número da página da subseção
\renewcommand{\cftsubsubsecafterpnum}{\vskip0pt} % sem espaço extra após o número da página da subseção

% Alinhando todos os níveis à esquerda
\setlength{\cftsecindent}{0em} % Ajuste esse valor se necessário
\setlength{\cftsubsecindent}{0em} % Ajuste esse valor se necessário
\setlength{\cftsubsubsecindent}{0em} % Ajuste esse valor se necessário
\setlength{\cftsecnumwidth}{3em} % Ajuste esse valor se necessário
\setlength{\cftsubsecnumwidth}{3em} % Ajuste esse valor se necessário
\setlength{\cftsubsubsecnumwidth}{3em} % Ajuste esse valor se necessário

% Remove o título "Conteúdo"
\renewcommand{\contentsname}{}
\addto\captionsportuguese{
  \renewcommand{\contentsname}{}
}

\begin{document}

% Capa
\begin{titlepage}
    \centering
    \begin{figure}
        \centering
        \includegraphics{imagens/ESAO.jpg}
        \label{fig:enter-label}
    \end{figure}
    \onehalfspacing
    \textbf{\Large ESCOLA DE APERFEIÇOAMENTO DE OFICIAIS} % Nome da instituição

    \vspace{2\baselineskip} % Dois espaços duplos

    \textbf{\myname} % Nome do autor

    \vspace{7\baselineskip} % Sete espaços duplos

    \textbf{\mytitle} % Título do trabalho

    \vspace{2\baselineskip} % Dois espaços duplos

    % \textbf{SUBTÍTULO DO TRABALHO} % Subtítulo do trabalho (se houver)

    \vspace{\baselineskip} % Um espaço duplo

    % \textbf{Vol. 1} % Volume (se houver)

    \vfill % Preenche o espaço vertical

    \vspace{5cm} % Espaço de 5cm da margem inferior

    \textbf{Campinas} % Cidade

    \vspace{\baselineskip} % Um espaço duplo

    \textbf{2023} % Ano de entrega
\end{titlepage}

\newgeometry{top=3cm,bottom=2cm,left=3cm,right=2cm} % Define novas margens

% Folha de rosto
\begin{titlepage}
    \centering
    \onehalfspacing
    \textbf{\Large ESCOLA DE APERFEIÇOAMENTO DE OFICIAIS} % Nome da instituição

    \vspace{3\baselineskip} % Três espaços duplos

    \textbf{\myname} % Nome do autor

    \vspace{4\baselineskip} % Quatro espaços duplos

    \textbf{\mytitle} % Título do trabalho

    \vspace{2\baselineskip} % Dois espaços duplos

    % \textbf{SUBTÍTULO DO TRABALHO} % Subtítulo do trabalho (se houver)

    \vspace{\baselineskip} % Um espaço duplo

    % \textbf{Vol. 1} % Volume (se houver)

    \vspace{4\baselineskip} % Quatro espaços duplos

    \begin{flushright}
    \begin{minipage}{0.5\textwidth}
        \singlespacing
        Trabalho de Conclusão de Curso apresentado à Escola de  Aperfeiçoamento de Oficiais como requisito parcial para a obtenção do grau especialização em Ciências Militares.

        \vspace{\baselineskip} % Um espaço simples

         Orientador: \myadvisor % Nome do orientador
    \end{minipage}
    \end{flushright}

    \vfill % Preenche o espaço vertical

    \vspace{4cm} % Espaço de 4cm da margem inferior

    \onehalfspacing
    \textbf{Campinas} % Cidade

    \vspace{\baselineskip} % Um espaço duplo

    \textbf{2023} % Ano de entrega
\end{titlepage}

\newpage % Iniciar uma nova página

\begin{titlepage}
    \centering
    \doublespacing

    \vspace*{3cm} % Espaçamento de 3cm do topo da página

    \textbf{\myname} % Nome do autor

    \vspace*{2\baselineskip} % Dois espaços duplos

    \textbf{\mytitle} % Título do trabalho

    \vspace*{\baselineskip} % Um espaço duplo

    % \textbf{SUBTÍTULO DO TRABALHO} % Subtítulo do trabalho

    \vspace*{\baselineskip} % Um espaço duplo

    \begin{flushright}
    \begin{minipage}{0.5\textwidth}

        \singlespacing
        Trabalho de Conclusão de Curso apresentado à Escola de  Aperfeiçoamento de Oficiais como requisito parcial para a obtenção do grau especialização em Ciências Militares.
    \end{minipage}
    \end{flushright}
    \doublespacing

    \vspace*{2\baselineskip} % Dois espaços duplos

    \begin{flushleft}
        Aprovado em 1 de março de 2023 % Data de aprovação
    \end{flushleft}

    \vspace*{2\baselineskip} % Dois espaços duplos

    % \textbf{Banca Examinadora} % Banca Examinadora
    COMISSÃO DE AVALIAÇÃO

    \begin{center}
        \makebox[10cm]{\hrulefill}
        \vspace{-0.5cm}
        \singlespacing
        Nome Completo e Posto \\
        Titulação \\
        Função na banca examinadora e instituição/OM de origem % Credenciais

        \vspace{4\baselineskip}

        \makebox[10cm]{\hrulefill}
        \vspace{-0.5cm}
        \singlespacing
        Nome Completo e Posto \\
        Titulação \\
        Função na banca examinadora e instituição/OM de origem % Credenciais
        \vspace{2\baselineskip}
    \end{center}
\end{titlepage}

\begin{titlepage}
    \centering
    \singlespacing
    \vspace*{3\baselineskip}

    \textbf{\Large RESUMO}

    \vspace{3\baselineskip}

    \singlespacing
    \leftskip1.25cm\rightskip1.25cm % ajusta a margem do parágrafo
    \lipsum[1] % gera um parágrafo fictício para representar o resumo
    \par

    \vspace{2\baselineskip}

    \textbf{Palavras-chave:} Palavra1. Palavra2. Palavra3. Palavra4.

\end{titlepage}

\begin{titlepage}
    \centering
    \singlespacing
    \vspace*{3\baselineskip}

    \textbf{\Large ABSTRACT}

    \vspace{3\baselineskip}

    \singlespacing
    \leftskip1.25cm\rightskip1.25cm % ajusta a margem do parágrafo
    \lipsum[1] % gera um parágrafo fictício para representar o resumo
    \par

    \vspace{2\baselineskip}

    \textbf{Key words} Palavra1. Palavra2. Palavra3. Palavra4.

\end{titlepage}

\begin{center}
    \vspace*{3cm}
    \Large\textbf{\MakeUppercase{Sumário}}
\end{center}
\vspace*{3\baselineskip}

\tableofcontents
\thispagestyle{empty}
\newpage

\setlength{\parindent}{1.25cm} % ajusta o recuo do parágrafo
\onehalfspacing % configura o espaçamento de 1,5 entre as linhas

\section{Primeira Seção}

\lipsum[1-2]

\section{Segunda Seção}

\lipsum[3-4]

\subsection{Sub seção}
\lipsum[5-6]

\subsubsection{Sub subseção}
\lipsum[5-6]

\end{document}
